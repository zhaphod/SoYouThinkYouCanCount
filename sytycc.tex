\documentclass{book}

\usepackage{mathpazo}
%\usepackage{mathptmx}
%\usepackage{concrete}
\usepackage{extramarks}

\usepackage{setspace}
\usepackage{amsmath, amsthm, amssymb}
\usepackage{epigraph}

%\usepackage{quotchap}
%\setbeamertemplate{navigation symbols}{}
%\usefonttheme[stillsansseriflarge]{serif}
\usepackage{etoolbox}


\newcommand{\enterProblemHeader}[1]{
    \nobreak\extramarks{}{Problem \arabic{#1} continued on next page\ldots}\nobreak{}
    \nobreak\extramarks{Problem \arabic{#1} (continued)}{Problem \arabic{#1} continued on next page\ldots}\nobreak{}
}

\newcommand{\exitProblemHeader}[1]{
    \nobreak\extramarks{Problem \arabic{#1} (continued)}{Problem \arabic{#1} continued on next page\ldots}\nobreak{}
    \stepcounter{#1}
    \nobreak\extramarks{Problem \arabic{#1}}{}\nobreak{}
}

\setcounter{secnumdepth}{0}
\newcounter{partCounter}
\newcounter{homeworkProblemCounter}
\setcounter{homeworkProblemCounter}{1}
\nobreak\extramarks{Problem \arabic{homeworkProblemCounter}}{}\nobreak{}

\newenvironment{homeworkProblem}{
    \section{Problem \arabic{homeworkProblemCounter}}
    \setcounter{partCounter}{1}
    \enterProblemHeader{homeworkProblemCounter}
}{
    \exitProblemHeader{homeworkProblemCounter}
}


\theoremstyle{plain}
%\newtheorem{thm}{Theorem}[chapter]
%\newtheorem{prop}[thm]{Proposition}
%\newtheorem{cor}[thm]{Corollary}
%\newtheorem{lem}[thm]{Lemma}
%\newtheorem{prob}[thm]{Problem}

% \epigraphsize{\small}% Default
\setlength\epigraphwidth{8cm}
\setlength\epigraphrule{0pt}

\makeatletter
\patchcmd{\epigraph}{\@epitext{#1}}{\itshape\@epitext{#1}}{}{}
\makeatother

\newcommand{\thetitle}{So You Think You Can Count}
\newcommand{\theversion}{1.2.2}

\title{\thetitle}
\author{Zhaphod Beeblebrox}

%\renewcommand{\@chapapp}{}% Not necessary...
\newenvironment{chapquote}[2][2em]
  {\setlength{\@tempdima}{#1}%
   \def\chapquote@author{#2}%
   \parshape 1 \@tempdima \dimexpr\textwidth-2\@tempdima\relax%
   \itshape}
  {\par\normalfont\hfill--\ \chapquote@author\hspace*{\@tempdima}\par\bigskip}
\makeatother

\begin{document}
%\frontmatter
%\maketitle

%\thispagestyle{empty}
%\begin{flushright}
%\vspace*{2.0in}
%
%\begin{spacing}{3}
%{\huge \thetitle}
%\end{spacing}
%
%\vspace{0.25in}
%
%Version \theversion
%
%\vfill
%
%\end{flushright}
\newcommand{\blankpage}{\thispagestyle{empty} \quad \newpage}

%\clearemptydoublepage
%\pagebreak
%\thispagestyle{empty}
%\vspace*{6in}

%--title page--------------------------------------------------
\pagebreak
\thispagestyle{empty}

\begin{flushright}
\vspace*{2.0in}

\begin{spacing}{3}
{\huge \thetitle}
\end{spacing}

\vspace{0.25in}

Version \theversion

\vspace{1in}


{\Large
Zhaphod Beeblebrox\\
}


\vspace{0.5in}

{\Large Megadodo Publication}

{\small Ursa Minor Beta}

%\includegraphics[width=1in]{figs/logo1.eps}
\vfill

\end{flushright}

\pagebreak
\begin{center}
\vspace*{2.0in}

{\Huge $\nu\epsilon$ $\pi\alpha\beta o \rho$}
\end{center}
%\blankpage
%\blankpage
\frontmatter
\chapter{Preface}
\epigraph{``Yes, but you need to learn your maths."\newline\newline
``I don't need to, really. I already know how to count to a hundred. And I'm sure I'll never need more than a hundred of anything.”}{-- \textup{Lisa Kleypas}, Love in the Afternoon}

Counting is mundane, easy, some thing that no one pays much attention to. Even kids as young as 3 can easily count. 
You may be wondering why on earth anyone would want to read a book on counting. However, in this mundane task of counting hides the enormous beauty that is hidden in our universe. And I would strive to convince you of the same.

\subsection{Why count?}
As human beings we learn to count from a very early age.A typical 2 year old will proudly
count to 10 for the coos and applause of adoring parents.Though many adults readily
claim ineptitude in mathematics, no one ever owns up to an inability to count. Counting
is one of our first tools, and it is time to appreciate its full mathematical power.The
physicist Ernst Mach even went so far as to say "There is no problem in all mathematics
that cannot besolved by direct counting"

\mainmatter
\chapter{Introduction}
%\begin{chapquote}{Lewis Carroll, \textit{Alice in Wonderland}}
%``Begin at the beginning,'' the King said, gravely, ``and go on till you
%come to an end; then stop.''
%\end{chapquote}
%\begin{savequote}
%``Begin at the beginning,'' the King said, gravely, ``and go on till you
%come to an end; then stop.''
%\qauthor{Lewis Carroll}
%\end{savequote}
%\begin{frame}
%\frametitle{A shaggy dog story in Palatino}


\epigraph{``Begin at the beginning," the King said gravely, ``and go on till you come to the end: then stop."}{--- \textup{Lewis Carroll}, Alice in Wonderland}

\epigraph{"We only think when confronted with a problem."}{-- \textup{John Dewey}}
			
\begin{homeworkProblem}
Show that every formula of $\mathcal{L}_P$ has the same number of left parentheses as it has of right parentheses.
\end{homeworkProblem}

\begin{homeworkProblem}
Consider the following number
\begin{quote}
\begin{center}
$12345678910111213...202122....979899100$
\end{center}
\end{quote}

Find the biggest number that can be formed by removing 100 digits from the above number.

\end{homeworkProblem}

%\end{frame}
\end{document}
